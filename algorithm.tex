%% What is an algorithm?
%%
%% Main idea: an algorithm is a finite sequence of instructions which, when
%% performed, reliably perform some task.
%%
%% Subpoints:
%%   1. finite: a finite algorithm is sufficient to perform an infinite number
%%      of tasks.
%%   2. sequence: a sequence can be expressed at various levels of granularity
%%      (low, medium, high, for example)
%%   3. task: one task may have many algorithms that perform it (though some
%%      have none at all [computability], and some have none that are efficient
%%      [complexity]).

Algorithm and computer are inseparable concepts.
In order to answer the question ``what is a computer?'', we must first answer the question ``what is an algorithm?'', in order to solidify what we want our computer to be able to do.
This chapter examines the properties that characterize our expectations for an algorithm.

We propose the definition of an algorithm as a finite sequence of instructions that, when followed, reliably performs a task.
With this definition in mind, we will be able to develop a concept of computer in the next chapter.

\section{Algorithms are finite}

Although it may run for an infinite number of steps, an algorithm has a finite description.
We can measure the size of an algorithm by counting, for example, the number of characters in its description.

\begin{example}
  The Python program below, which prints the first ten natural numbers, has 32 characters (including the line break character).
  \begin{lstlisting}[language=Python]
    for i in range(10):
        print(i)
  \end{lstlisting}
\end{example}

\begin{example}
  The Python program below, which repeatedly prints the number seven, has 38 characters (including the line break character).
  \begin{lstlisting}[language=Python]
    while True:
        print(7)
  \end{lstlisting}
  Although this algorithm is finite, it runs forever when executed by a computer.
\end{example}

Infinite programs do not match our intuitive notion of algorithms.
How would one express an infinite algorithm?
How would one provide an infinite algorithm to a computer?

\section{Algorithms are sequential}

An algorithm is a sequence of discrete instructions.
A sequence, not a set, is crucial: ``sweep, then mop'' is different from ``mop, then sweep''.
For our purposes, such a sequence can be expressed at three levels of granularity: high-level, implementation-level, and formal descriptions.

A high-level description of an algorithm is an English language description which relies on our shared understanding of how to write computer programs.

\begin{example}
  Given a polynomial, find the roots of the polynomial.

  This description is probably at too high a level to be considered a high-level description of an algorithm; the procedure for finding the roots of a polynomial is not immediately obvious, nor can it be inferred from context.
\end{example}

\begin{example}
  Given a list of integers, sort the list.

  This algorithm is probably acceptable as a high-level description of an algorithm in most situations, because most readers know some procedure for sorting a integers.
\end{example}

\begin{example}
  The following is a high-level description of an algorithm that decides whether two vertices in a graph are connected.
  Given a graph and two vertices $s$ and $t$ in the graph, initially mark $s$, then repeatedly mark unmarked neighbors of any marked nodes, until there are no more such neighbors.
  The vertices $s$ and $t$ are connected if and only if $t$ becomes marked.
\end{example}

An implementation-level description is a mostly English language description of how the machine executing the algorithm should act, relying on our programming experience to fill in the details of how exactly the machine should act.

\begin{example}
\end{example}

A formal description, with respect to a particular machine, is the exact program that, when provided exactly as written to that machine, performs the task.
This is the C program, the Python program, and later in this book, the Turing machine program.

\begin{example}
  This is the formal description, in the Python programming language, of an algorithm that performs a depth-first search on a tree starting from \verb|current_node|, printing each node encountered during the search.
  \begin{lstlisting}[language=Python]
    def depth_first_search(current_node):
        print(current_node)
        for child in current_node.children:
            depth_first_search(child)
  \end{lstlisting}
\end{example}

\section{Algorithms perform a task}

An algorithm, when followed precisely, reliably performs a task.
A task may have many algorithms that perform it.
The following three Python programs perform the same task: printing the number two.
\begin{lstlisting}[language=Python]
print(2)

print(3 - 1)

print(4 - 2)
\end{lstlisting}
Extrapolating from these examples, there are an infinite number of programs that perform this task.
In fact, if a task admits an algorithm at all, then it admits an infinite number of algorithms.
However, not all tasks have algorithms that perform them (this is Computability Theory, \autoref{prt:computability}), and each algorithm for a task may use a different amount of resources like time, space, or randomness (this is Complexity Theory, \autoref{prt:complexity}).
