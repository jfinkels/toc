Just as there are uncountable sets, there are undecidable languages.

\begin{theorem}
  The language $K$ defined by
  $$
  K = \{ \langle M \rangle | M \text{ accepts } \langle M \rangle \}
  $$
  is undecidable.
\end{theorem}
\begin{proof}
  This is a proof by contradiction.
  Assume that $K$ is decidable, and suppose $D$ is the Turing machine that decides it.
  Let $N$ be the Turing machine that runs $D$ and outputs the opposite of whatever $D$ outputs, i.e. $N(x) = \lnot D(x)$ for each binary string $x$.
  There are two cases to consider: whether $\langle N \rangle$ is in $K$ or not.

  If $\langle N \rangle$ is in $K$, then by definition $N$ accepts $\langle N \rangle$, which means $D$ rejects $\langle N \rangle$ by construction.
  Since $D$ decides the language $K$ and $D$ rejects $\langle N \rangle$, we know that $\langle N \rangle$ must not be in $K$.
  This is a contradiction with our hypothesis.

  Conversely, if $\langle N \rangle$ is not in $K$, then $N$ rejects $\langle N \rangle$, which means $D$ accepts $\langle N \rangle$.
  Thus $\langle N \rangle$ must be in $K$, again a contradiction with our hypothesis.

  In either case, we have a contradiction, thus we conclude that $K$ must not be decidable.
\end{proof}

With this first undecidable language, we can prove the undecidability of some other more sensible languages.

\begin{theorem}
  The language $A$ defined by
  $$
  A = \{ \langle M, x \rangle | M \text{ accepts } x\}
  $$
  is undecidable.
\end{theorem}
\begin{proof}
  This is a direct proof.
  We show a computable many-one reduction from $K$ to $A$.
  The reduction is $\langle M \rangle \mapsto \langle M, \langle M \rangle \rangle$.
  This function is computable because it essentially duplicates its input.
  If $\langle M \rangle$ is in $K$, then $M$ accepts $\langle M \rangle$, so $\langle M, \langle M \rangle \rangle$ is in $A$.
  Conversely, if $\langle M \rangle$ is not in $K$, then $M$ rejects $\langle M \rangle$, so $\langle M, \langle M \rangle \rangle$ is not in $A$.
  Thus we have a computable Turing reduction from $K$ to $A$.
  Since $K$ is undecidable, $A$ is undecidable as well by \autoref{lem:manyone}.
\end{proof}

\section{The halting problem}

The language $H$ in the following theorem when viewed as a computational problem is known as the \emph{halting problem}.
It is the formalization of the problem of programmatically deciding whether another computer program either terminates or loops forever.
This theorem indicates that no such procedure exists for the general case of deciding whether an arbitrary computer program terminates.

\begin{theorem}
  The language $H$ defined by
  $$
  H = \{ \langle M, x \rangle | M \text{ halts on input } x\}
  $$
  is undecidable.
\end{theorem}
\begin{proof}
  This is a direct proof.
  We show a computable Turing reduction from $A$ to $H$.
  The reduction is the function, $\langle M, x \rangle \mapsto \langle M', x \rangle$, where $M'$ is the Turing machine that simulates $M$, but when $M$ rejects, $M'$ loops forever.

  If $\langle M, x \rangle$ is in $A$, then $M$ accepts $x$, and so does $M'$.
  Specifically, $M'$ halts on $x$, and thus $\langle M', x \rangle$ is in $H$.

  Conversely, if $\langle M, x \rangle$ is not in $A$, then $M$ rejects $x$, so $M'$ loops forever on $x$.
  This means $M'$ does not halt on $x$, thus $\langle M', x \rangle$ is not in $H$.
  Thus we have a computable Turing reduction from $A$ to $H$.
  Since $A$ is undecidable, $H$ is undecidable as well by \autoref{lem:manyone}.
\end{proof}

\begin{theorem}
  The language $J$ defined by
  $$
  J = \{ \langle M \rangle | \text{ there is an } x \text{ such that } M \text{ halts on } x\}
  $$
  is undecidable.
\end{theorem}
\begin{proof}
  This is a direct proof.
  We show a computable Turing reduction from $H$ to $J$.
  The reduction is the function, $\langle M, x \rangle \mapsto \langle M' \rangle$, where $M'$ is the Turing machine that simulates $M$ when it receives $x$ as input but loops forever on any input besides $x$.

  If $\langle M, x \rangle$ is in $H$, then $M$ halts on input $x$, and so does $M'$ by construction.
  Thus there is an $x$ such that $M'$ halts, so $\langle M' \rangle$ is in $J$.

  Conversely, if $\langle M, x \rangle$ is not in $H$, then $M$ does not halt on input $x$, so $M'$ also does not halt on input $x$.
  Since $M'$ does not halt on input $x$ and does not halt on any other input by construction, $M'$ does not halt for any $x$, thus $\langle M' \rangle$ is not in $J$.
  Thus we have a computable Turing reduction from $H$ to $J$.
  Since $H$ is undecidable, $J$ is undecidable as well by \autoref{lem:manyone}.
\end{proof}
