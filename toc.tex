%% toc.tex - Theory of Computation: An Introduction
%%
%% Copyright 2014 Jeffrey Finkelstein.
%%
%% This LaTeX markup document is made available under the terms of the Creative
%% Commons Attribution-ShareAlike 4.0 International License,
%% https://creativecommons.org/licenses/by-sa/4.0/.
\documentclass{book}

\usepackage{amsmath}
\usepackage{amssymb}
%% This must come before hyperref.
\usepackage{amsthm}
%% This is strongly recommended by biblatex.
\usepackage[english]{babel}
\usepackage[backend=biber]{biblatex}
\usepackage[T1]{fontenc}
\usepackage{lipsum}
\usepackage{listings}
%% This must come before csquotes.
\usepackage[utf8]{inputenc}
\usepackage{cfr-lm}
%% This is strongly recommended by biblatex.
\usepackage{csquotes}
%% This must come before hyperref.
\usepackage{thmtools}
%% This must come before complexity.
\usepackage{hyperref}
\usepackage{complexity}
\usepackage[firstpage]{draftwatermark}

\usepackage[final]{microtype}
\usepackage{textcomp}

%% Set the amount by which certain characters protrude into the margins.
%%
%% \LoadMicrotypeFile{cmr}
%%
%%     This command forces the built-in protrusion settings for the Computer
%%     Modern Roman (cmr) font family to become available at this point, so
%%     that we can override these settings on the next line.
%%
%% \SetProtrusion
%%
%%     This instructs the microtype package that we are going to modify the
%%     protrusion settings.
%%
%% [load=cmr-T1]
%%
%%     Loads the Type 1 (T1) encoding of the font family named `cmr`, thereby
%%     setting the default protrusion values for all the characters. This is
%%     only possible after the \LoadMicrotypeFile{cmr} command.
%%
%% {encoding=T1, family=cmr}
%%
%%     Indicates that we are going to modify the protrusion values for the T1
%%     encoding of the `cmr` font family.
%%
%% \textquotedblright = {,1000} (and similar commands)
%%
%%     Force the character given by \textquotedblright to have default
%%     protrusion on the left margin (given by an empty string before the
%%     comma) and full protrusion (that is, protrusion value 1000) on the right
%%     margin.
\LoadMicrotypeFile{cmr}
\SetProtrusion
    [load=cmr-T1]
    {encoding=T1, family=cmr}
    {
      \textquotedblright = {,1000},
      \textquotedblleft = {1000,},
      {'} = {,1000},
      {,} = {,1000},
      {:} = {,1000},
      {;} = {,1000},
      {.} = {,1000}
    }

%% Set the ``work-in-progress'' watermark for the first page.
\SetWatermarkLightness{0.9}
\SetWatermarkText{Work-in-progress}
\SetWatermarkFontSize{3.5cm}

%% Set the title and author of the PDF file.
\hypersetup{
  pdftitle={Theory of Computation: An Introduction},
  pdfauthor={Jeffrey Finkelstein},
  pagebackref=true,
}

%% Declare the bibliography file.
\addbibresource{document.bib}

\lstset{
  basicstyle=\ttfamily, %% monospace fonts in code listings
  %upquote=true, %% make single quotes straight, not curly
  showstringspaces=false  %% don't show underscores instead of spaces
}

%% Declare theorem-like environments.
\declaretheorem[numberwithin=chapter]{theorem}
\declaretheorem[numberlike=theorem]{lemma}
\declaretheorem[numberlike=theorem]{corollary}
\declaretheorem[numberlike=theorem, style=definition]{definition}
\declaretheorem[numberlike=theorem, style=definition]{example}

%% Custom commands are declared here.
\newcommand{\email}[1]{\textlangle\href{mailto:#1}{\nolinkurl{#1}}\textrangle}
\newcommand{\todo}[1]{\textbf{TODO #1}}

%% Redefine the footnote environment so it has no reference and no number.
\long\def\symbolfootnote#1{\begingroup%
\def\thefootnote{\fnsymbol{footnote}}\footnotetext{#1}\endgroup}

% store the title in two places
\makeatletter
\def\title#1{\gdef\@title{#1}\gdef\THETITLE{#1}}
\makeatother

%% Define the author, title, and date for the document.
\author{Jeffrey Finkelstein}
\title{Theory of Computation: An Introduction}

\newcommand{\makehalftitle}{
  {\centering \Large
    \vspace*{5em}
    \textsc{\THETITLE}
  }
}

\begin{document}

\frontmatter

% Half title page goes here
\makehalftitle

% Empty page

% Title page
\maketitle

% Copyright and ISBN information
{\small
Copyright 2014 Jeffrey Finkelstein.

{\centering \includegraphics[width=88px,height=31px]{ccbysa}}

This work is licensed under the Creative Commons Attribution-ShareAlike 4.0
International License. To view a copy of this license, visit
\url{http://creativecommons.org/licenses/by-sa/4.0/} or send a letter to
Creative Commons, 444 Castro Street, Suite 900, Mountain View, California,
94041, USA.

The \LaTeX{} source code for this book is published under the same license, and is available at \url{https://github.com/jfinkels/toc}.
}


% Dedication goes here
\newpage
\input{dedication}

\tableofcontents

\chapter{Preface}
% Specify the intended audience.
%
% State that this is a work-in-progress, and that contributions are welcome.
\lipsum


\chapter{Introduction}
%% For further goals, see
%% https://www.acm.org/education/curricula/ComputerScience2008.pdf
As a student of computer science, you have written many computer programs and can analyze and construct both data structures and algorithms that employ those data structures.
In order to fully understand what you \emph{can} do, however, you must understand what you \emph{cannot} do; this is the ultimate goal of theoretical computer science.
This book provides you with an introduction to the fundamental limits of computation using the tools of mathematics.
This book answers three questions.
\begin{enumerate}
\item What is a computer (and what is not a computer)?
\item What problems can a computer solve (and what can it never solve)?
\item What problems can a computer solve efficiently (and what can it never solve efficiently)?
\end{enumerate}

We will discover the answers in Parts I, II, and III, respectively.
\begin{enumerate}
\item The Turing machine captures any reasonable notion of computation.
\item Some problems that are simple to express (but allow for the possibility of self-reference) cannot be solved by any computer.
\item A computer cannot quickly discover a proof for a mathematical theorem, and cannot solve any other problem that captures the difficulty of that problem.
\end{enumerate}
Knowing the fundamental limits of computation allows you to quickly decide whether to develop an algorithm for a problem, or whether to take a different approach (like developing an approximation algorithm instead of an exact algorithm, or simply spending your person-hours in some other endeavor).
Once you have a basic understanding of what computers can and cannot do (and how to prove it), there are many, many further topics to explore in theoretical computer science.
%% For more information, see...


\newpage
\input{dedication}

% Second half title page goes here
\newpage
\makehalftitle

\mainmatter

\part{Models of computation} \label{prt:models}
%% Main message: The Turing machine is a computational model which can
%% implement any ``algorithm''.
%%
Our intuitive notion of ``algorithm'' and ``computer'' is insufficient for proving mathematical statements about correctness or efficiency of computer programs.
A theoretical computer scientist needs a formal definition of a computer that can implement precisely defined algorithms, so that our claims can be backed up by mathematical proofs.
This part of the book defines an algorithm, proposes several models of computing devices, and presents the thesis that one of these models of computation is sufficient for implementing any algorithm.

We define an algorithm as a finite sequence of instructions, suggest a Turing machine as the most basic model of computation for executing any algorithm, and offer evidence that, although simple, the Turing machine is powerful enough to simulate any other reasonable model of computation.
Proving which problems are solvable (\autoref{prt:computability}) and which are solvable efficiently (\autoref{prt:complexity}) is only possible with the Turing machine clearly defined.
Although the Turing machine is sufficient to describe any computation, there are many, many other models of computation.
%% For more information, see...


\chapter{What is an algorithm?}
%% What is an algorithm?
%%
%% Main idea: an algorithm is a finite sequence of instructions which, when
%% performed, reliably perform some task.
%%
%% Subpoints:
%%   1. finite: a finite algorithm is sufficient to perform an infinite number
%%      of tasks.
%%   2. sequence: a sequence can be expressed at various levels of granularity
%%      (low, medium, high, for example)
%%   3. task: one task may have many algorithms that perform it (though some
%%      have none at all [computability], and some have none that are efficient
%%      [complexity]).
\lipsum

\input{algorithm-ex}

\input{algorithm-ex}

\chapter{What is a computer?}
%% Main message: a computer consists of input, output, memory, and some form of
%% state control.
\section{Computers have input}
\section{Computers have output}
\section{Computers have memory}
\section{Computers have state control}

\input{computer-ex}

\chapter{The Turing Machine model}
%% Main message: a Turing machine is a model of computer that can implement
%% algorithms.
\lipsum

\input{turingmachine-ex}

\chapter{Simulations}
%% Main message: a Turing machine can simulate other models of computation.
%%
\lipsum

\input{simulations-ex}

\chapter{The Church--Turing Thesis}
%% Main message: The set of ``algorithms'', in our intuitive sense, equals the
%% set of all Turing machines.
\lipsum

\input{churchturing-ex}

\part{Computability} \label{prt:computability}
\input{computabilityintro}

\chapter{Countable sets}
\begin{definition}
  A set $S$ is \emph{countably infinite} if there is a bijection from $\mathbb{N}$ to $S$ (or vice versa, since bijections are invertible).
  A set is \emph{uncountably infinite} if no such bijection exists.
\end{definition}

The natural numbers are countably infinite as witnessed by the identity function.
The integers are countably infinite as well.

\begin{theorem}
  The set of integers $\mathbb{Z}$ is countably infinite.
\end{theorem}
\begin{proof}
  The required bijection is the function $f(n) = (-1)^{p} \lceil n / 2 \rceil$, where $p$ is the parity of $n$, that is, $p = 1$ if $n$ is odd and $p = 0$ if $n$ is even.
  When written as an ordered sequence, this yields
  $$
  (0, -1, +1, -2, +2, \dotsc)
  $$
\end{proof}

The set of all pairs of natural numbers is countably infinite as well; we use a special function to define an order for them.

\begin{definition}
  Suppose $S$ and $T$ are two countably infinite sets as witnessed by the functions $s$ and $t$, respectively.
  The \emph{dovetail function} $d \colon \mathbb{N} \to (S \times T)$ is defined as described in the following diagram.
  \ldots
\end{definition}

\begin{lemma}
  Suppose $S$ and $T$ are two countably infinite sets.
  The dovetail function on $S$ and $T$ is a bijection.
\end{lemma}
\begin{proof}
  One can prove this directly by using the formula for the dovetail function, $d(n) = ...$
\end{proof}

The dovetail function is a bijection from the natural numbers to the Cartesian product of two countably infinite sets, which means the Cartesian product itself is countably infinite.

\begin{theorem}
  The Cartesian product of two countably infinite sets is countably infinite.
\end{theorem}

\begin{corollary}
  The set of rational number $\mathbb{Q}$ is countably infinite.
\end{corollary}
\begin{proof}
  The set of rational numbers is the set of quotients of integers; each rational number is of the form $m / n$, where $m$ and $n$ are integers.
  There is a bijection from the set of pairs of integers to the set of rational numbers, namely the function $(m, n) \mapsto m / n$.
  Since bijections compose, it now suffices to prove that the set of pairs of integers is countably infinite.
  Since the set of integers is countably infinite \autoref{thm:integerscountable} and the Cartesian product of two countably infinite sets is countably infinite \autoref{thm:cartesiancountable}, the set of pairs of integers is countably infinite.
  Now since bijections compose and there is both a bijection from $\mathbb{N}$ to $\mathbb{Z} \times \mathbb{Z}$ and a bijection from $\mathbb{Z} \times \mathbb{Z}$ to $\mathbb{Q}$, we have therefore shown a bijection from $\mathbb{N} to \mathbb{Q}$.
  Therefore $\mathbb{Q}$ is countably infinite.
\end{proof}

The theorem above can be extended to any arbitrary number of Cartesian products.

\begin{theorem}
  Suppose $n$ is a natural number and $\{S_0, \dotsc, S_{n - 1}\}$ is a collection of $n$ countably infinite sets.
  Then $S_0 \times \dotsb \times S_{n - 1}$ is countably infinite.
\end{theorem}
\begin{proof}
  This is a proof by mathematical induction on $n$.
  For the base case, if $n$ is zero, then the collection of sets is empty, so there is a trivial bijection to it (the empty function).
  For the inductive case, suppose that the proposition is true for any collection of $n - 1$ countably infinite sets.
  Since the Cartesian product of $n$ sets can be written as $(S_0 \times \dotsb \times S_{n - 2}) \times S_{n - 1}$, we can use the inductive hypothesis to state that $S_0 \times \dotsb \times S_{n - 2}$ is countably infinite.
  Now we have the Cartesian product of $S_0 \times \dotsb S_{n - 2}$ and $S_{n - 1}$, two countably infinite sets.
  By \autoref{thm:productcountable}, this product is countably infinite.
  Therefore we have shown by induction that the proposition is true for all natural numbers $n$.
\end{proof}

\input{countable-ex}

\chapter{Uncountable sets}
\input{uncountable}
\input{uncountable-ex}

\chapter{Recognizable languages}
\input{recognizable}
\input{recognizable-ex}

\chapter{The halting problem}
Just as there are uncountable sets, there are undecidable languages.

\begin{theorem}
  The language $K$ defined by
  $$
  K = \{ \langle M \rangle | M \text{ accepts } \langle M \rangle \}
  $$
  is undecidable.
\end{theorem}
\begin{proof}
  This is a proof by contradiction.
  Assume that $K$ is decidable, and suppose $D$ is the Turing machine that decides it.
  Let $N$ be the Turing machine that runs $D$ and outputs the opposite of whatever $D$ outputs, i.e. $N(x) = \lnot D(x)$ for each binary string $x$.
  There are two cases to consider: whether $\langle N \rangle$ is in $K$ or not.

  If $\langle N \rangle$ is in $K$, then by definition $N$ accepts $\langle N \rangle$, which means $D$ rejects $\langle N \rangle$ by construction.
  Since $D$ decides the language $K$ and $D$ rejects $\langle N \rangle$, we know that $\langle N \rangle$ must not be in $K$.
  This is a contradiction with our hypothesis.

  Conversely, if $\langle N \rangle$ is not in $K$, then $N$ rejects $\langle N \rangle$, which means $D$ accepts $\langle N \rangle$.
  Thus $\langle N \rangle$ must be in $K$, again a contradiction with our hypothesis.

  In either case, we have a contradiction, thus we conclude that $K$ must not be decidable.
\end{proof}

With this first undecidable language, we can prove the undecidability of some other more sensible languages.

\begin{theorem}
  The language $A$ defined by
  $$
  A = \{ \langle M, x \rangle | M \text{ accepts } x\}
  $$
  is undecidable.
\end{theorem}
\begin{proof}
  This is a direct proof.
  We show a computable many-one reduction from $K$ to $A$.
  The reduction is $\langle M \rangle \mapsto \langle M, \langle M \rangle \rangle$.
  This function is computable because it essentially duplicates its input.
  If $\langle M \rangle$ is in $K$, then $M$ accepts $\langle M \rangle$, so $\langle M, \langle M \rangle \rangle$ is in $A$.
  Conversely, if $\langle M \rangle$ is not in $K$, then $M$ rejects $\langle M \rangle$, so $\langle M, \langle M \rangle \rangle$ is not in $A$.
  Thus we have a computable Turing reduction from $K$ to $A$.
  Since $K$ is undecidable, $A$ is undecidable as well by \autoref{lem:manyone}.
\end{proof}

\section{The halting problem}

The language $H$ in the following theorem when viewed as a computational problem is known as the \emph{halting problem}.
It is the formalization of the problem of programmatically deciding whether another computer program either terminates or loops forever.
This theorem indicates that no such procedure exists for the general case of deciding whether an arbitrary computer program terminates.

\begin{theorem}
  The language $H$ defined by
  $$
  H = \{ \langle M, x \rangle | M \text{ halts on input } x\}
  $$
  is undecidable.
\end{theorem}
\begin{proof}
  This is a direct proof.
  We show a computable Turing reduction from $A$ to $H$.
  The reduction is the function, $\langle M, x \rangle \mapsto \langle M', x \rangle$, where $M'$ is the Turing machine that simulates $M$, but when $M$ rejects, $M'$ loops forever.

  If $\langle M, x \rangle$ is in $A$, then $M$ accepts $x$, and so does $M'$.
  Specifically, $M'$ halts on $x$, and thus $\langle M', x \rangle$ is in $H$.

  Conversely, if $\langle M, x \rangle$ is not in $A$, then $M$ rejects $x$, so $M'$ loops forever on $x$.
  This means $M'$ does not halt on $x$, thus $\langle M', x \rangle$ is not in $H$.
  Thus we have a computable Turing reduction from $A$ to $H$.
  Since $A$ is undecidable, $H$ is undecidable as well by \autoref{lem:manyone}.
\end{proof}

\begin{theorem}
  The language $J$ defined by
  $$
  J = \{ \langle M \rangle | \text{ there is an } x \text{ such that } M \text{ halts on } x\}
  $$
  is undecidable.
\end{theorem}
\begin{proof}
  This is a direct proof.
  We show a computable Turing reduction from $H$ to $J$.
  The reduction is the function, $\langle M, x \rangle \mapsto \langle M' \rangle$, where $M'$ is the Turing machine that simulates $M$ when it receives $x$ as input but loops forever on any input besides $x$.

  If $\langle M, x \rangle$ is in $H$, then $M$ halts on input $x$, and so does $M'$ by construction.
  Thus there is an $x$ such that $M'$ halts, so $\langle M' \rangle$ is in $J$.

  Conversely, if $\langle M, x \rangle$ is not in $H$, then $M$ does not halt on input $x$, so $M'$ also does not halt on input $x$.
  Since $M'$ does not halt on input $x$ and does not halt on any other input by construction, $M'$ does not halt for any $x$, thus $\langle M' \rangle$ is not in $J$.
  Thus we have a computable Turing reduction from $H$ to $J$.
  Since $H$ is undecidable, $J$ is undecidable as well by \autoref{lem:manyone}.
\end{proof}

\input{halting-ex}

\chapter{Decidability}
\input{decidable}
\input{decidable-ex}

\chapter{Computable functions}
\begin{definition}
  A function on binary strings is \emph{computable} if there is a Turing machine $M$ such that for each input $x$, the machine $M$ halts with $f(x)$ on its tape.
\end{definition}

Some functions are not computable.

\begin{theorem}
  Suppose $L$ is an undecidable language.
  The characteristic function $\chi_L$ is uncomputable.
\end{theorem}
\begin{proof}
  This is a proof by contradiction.
  If $\chi_L$ is computable, then $L$ can be decided by the Turing machine that, on input $x$, computes $\chi_L(x)$ and accepts exactly when $\chi_L(x) = 1$.
  If $x$ is in $L$, then $\chi_L(x) = 1$, so the machine accepts.
  Conversely, if $x$ is not in $L$, then $\chi_L(x) = 0$, so the machine rejects.
  Therefore $L$ is decidable---a contradiction with the hypothesis.
\end{proof}

For example, the characteristic function of the halting problem, $\chi_H$, is uncomputable.

\section{Kolmogorov complexity}

The Kolmogorov complexity of a binary string is the length of the smallest Turing machine that outputs it, where ``small'' is measured by the number of bits in the encoding of the machine.

\begin{definition}
  The \emph{Kolmogorov complexity} function, denoted $K$, is defined by
  $$
  K(x) = \min \{ |\langle M \rangle| | M \text{ halts on input } \lambda \text{ with } x \text{ on its tape}\},
  $$
  for all binary strings $x$.
\end{definition}

Kolmogorov complexity is a notion of the amount of ``information'' present in a binary string.
If a string very little information, then it can be described by a very short program.
For example, a string of fifty ones can be expressed in Python as follows.
\begin{lstlisting}[language=Python]
  [1] * 50
\end{lstlisting}
However, a string of fifty bits generated from a random number generator is more difficult to describe succinctly:
%% From:  n=50; for i in `seq $n`; do shuf -i 0-1 -n 1; done | paste -s -d "
$$
01000001110000010110111010000001100000010000000110
$$

\begin{definition}
  A binary string $x$ is called \emph{incompressible} if $K(x) \geq |x|$.
\end{definition}

These kinds of strings exist, and there are an infinite number of them.

\begin{theorem}
  For each natural number $n$, there is an incompressible string of length $n$.
\end{theorem}
\begin{proof}
  This is a proof by counting argument.
  Let $n$ be a natural number.

  %% Consider the \emph{evaluation function}, the function $\langle M \rangle \mapsto x$, where $\langle M \rangle$ is of length at most $n - 1$, the string $x$ is of length $n$, and $M$ halts on input $\lambda$ with $x$ on its tape.
  %% We will show that no function $f \colon \mathcal{M}^{\leq n - 1} \to \{0, 1\}^n$ is surjective, which in turn shows that the evaluation function is not surjective.

  There are $2^i$ binary strings of length $i$, so the total number of binary strings of length at most $n - 1$ is
  $$
  \sum_{i = 0}^{n - 1} 2^i,
  $$
  which equals $2^n - 1$.
  This is the number of Turing machine descriptions of length at most $n - 1$.
  On the other hand, the number of binary strings of length $n$ is $2^n$.
  Since $2^n - 1 < 2^n$, any function from the set of Turing machine descriptions of length at most $n - 1$ to the set of strings of length $n$ cannot be surjective.
  Specifically, the function that maps $\langle M \rangle$ to $x$ if $M$ halts on input $\lambda$ with $x$ on its tape, where $|\langle M \rangle|$ is of length at most $n - 1$ and $x$ is of length $n$, is not surjective.
  Thus there is a string such that for each Turing machine $M$ whose description is of length at most $n - 1$, the machine $M$ does not output $x$.
  That means that the description of any Turing machine that outputs $x$ must have length at least $n$, or in other words, $K(x) \geq n$.
  Therefore there is an incompressible string of length $n$.

  %% A string $x$ of length $n$ is incompressible if $K(x) \geq n$, which means by definition
  %% $$
  %% \min \{ |\langle M \rangle| | M \text{ halts on input } \lambda \text{ with } x \text{ on its tape}\} \geq n,
  %% $$
  %% or equivalently, for each Turing machine $M$ with $|\langle M \rangle| < n$, the machine $M$ either does not halt on input $\lambda$ or halts with something other than $x$ on its tape.

  % we will show that the set of strings of length $n$ is strictly larger in cardinality pthan the set of Turing machines whose description is of length at most $n - 1$.

  % We count the number of strings of length $n$ and compare that with the number of Turing machine description
\end{proof}

\begin{theorem}
  $K$ is uncomputable.
\end{theorem}
\begin{proof}
  This is a proof by contradiction.
  Assume $K$ is computable, so there is a Turing machine $C$ that always halts with $K(x)$ on its tape on input $x$.

  For each natural number $n$, we can construct a Turing machine $C_n$ defined as follows.
  Ignoring its input, for each binary string $x$ in lexicographic order, if $C(x) \geq n$, then halt and output $x$.
  This Turing machine always halts because $C$ always halts by assumption and there are incompressible strings of each length (\autoref{thm:incompressible}), so the machine must halt after it reaches the incompressible string of length $n$.
  The machine $C_n$ on input $\lambda$ halts with a string $x$ on its tape such that $x$ is the lexicographically first string satisfying $K(x) \geq n$.

  Let $y$ be the lexicographically first string with $K(y) \geq n$.
  Since $C_n$ outputs $y$, we have $K(y) \leq |\langle C_n \rangle|$.
  Now, using $c$, $c'$, $c''$, and $d$ to denote constants,
  \begin{align*}
  K(y) & \leq |\langle C_n \rangle| \\
       & = |\langle C \rangle| + |n| + c \\
       & = |\langle C \rangle| + \log n + c' + c \\
       & = c'' + \log n + c' + c \\
       & = \log n + d
  \end{align*}
  where $c$, $c'$, and $c''$ are constants that depend on the encoding of the Turing machines $C_n$ and $C$ and the natural number $n$, and $d = c'' + c' + c$.
  The equation on the second line holds because the description for $C_n$ includes the description of $C$ and the description of $n$.
  The equation on the third line holds because a natural number can be encoded in binary using about $\log n$ bits.
  The fourth line uses the fact that the description of $C$ is a constant independent of $n$ (note that it was quantified before $n$ in this proof).

  Combining our two inequalities, we have $n \leq K(y) \leq \log n + d$, so by the transitive property, $n \leq \log n + d$.
  For sufficiently large $n$, we know $n > \log n + d$ by \autoref{lem:ngtlogn}, so this is a contradiction.
  Therefore, $K$ must not be computable.
\end{proof}

\begin{corollary}
  The following languages are undecidable.
  \begin{enumerate}
  \item The set of incompressible strings.
  \item The set of pairs $(x, k)$ such that $K(x) \geq k$.
  \end{enumerate}
\end{corollary}
\begin{proof}
  The first proof is by contradiction.
  Assume $K$ is computable.
  We can decide whether a string $x$ is incompressible by computing $K(x)$ and then comparing that with $|x|$, accepting exactly when $K(x) \geq |x|$.
  The second proof is a direct proof by a computable many-one reduction from the first language to the second.
  The reduction is $x \mapsto (x, |x|)$, computable because the Turing machine computing the reduction copies the input to its output tape and writes the length of its input on its output tape.
  Since $x$ is incompressible if and only if $K(x) \geq |x|$, the reduction is correct.
\end{proof}

\input{functions-ex}

\part{Complexity} \label{prt:complexity}
\input{complexityintro}

\chapter{Time}
\input{time}
\input{time-ex}

\chapter{Space}
\input{space}
\input{space-ex}

\chapter{Reductions}
\input{reductions}
\input{reductions-ex}

\chapter{P vs NP}
\input{pvsnp}
\input{pvsnp-ex}

\chapter{Randomness}
\input{randomness}
\input{randomness-ex}

\appendix
\chapter{Symbols}
\input{symbols}

\backmatter

\chapter{Bibliography}
\input{bibliography}

\chapter{Index}
\input{index}

\end{document}
