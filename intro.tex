%% For further goals, see
%% https://www.acm.org/education/curricula/ComputerScience2008.pdf
As a student of computer science, you have written many computer programs and can analyze and construct both data structures and algorithms that employ those data structures.
In order to fully understand what you \emph{can} do, however, you must understand what you \emph{cannot} do; this is the ultimate goal of theoretical computer science.
This book provides you with an introduction to the fundamental limits of computation using the tools of mathematics.
This book answers three questions.
\begin{enumerate}
\item What is a computer (and what is not a computer)?
\item What problems can a computer solve (and what can it never solve)?
\item What problems can a computer solve efficiently (and what can it never solve efficiently)?
\end{enumerate}

We will discover the answers in Parts I, II, and III, respectively.
\begin{enumerate}
\item The Turing machine captures any reasonable notion of computation.
\item Some problems that are simple to express (but allow for the possibility of self-reference) cannot be solved by any computer.
\item A computer cannot quickly discover a proof for a mathematical theorem, and cannot solve any other problem that captures the difficulty of that problem.
\end{enumerate}
Knowing the fundamental limits of computation allows you to quickly decide whether to develop an algorithm for a problem, or whether to take a different approach (like developing an approximation algorithm instead of an exact algorithm, or simply spending your person-hours in some other endeavor).
Once you have a basic understanding of what computers can and cannot do (and how to prove it), there are many, many further topics to explore in theoretical computer science.
%% For more information, see...
