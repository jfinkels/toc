%% Context: There are countless other models of computation: variants of Turing
%% machines, random access machines, register machines, pointer machines,
%% nondeterministic machines, probabilistic machines, etc.
%%
%% Need: In order to prove whether problems are solvable or not solvable by
%% computers, it is important to show that the model of computation we have
%% chosen is at least as powerful as any other model of computation.
%%
%% Task: Our task is to determine if the Turing machine is at least as powerful
%% as any other model of computation, so that when we prove a problem is not
%% solvable by a Turing machine, we are simultaneously proving that it is not
%% solvable by any other computer.
%%
%% Object: The Turing machine can simulate any model of computation you can
%% come up with, with some time and space overhead.
%%
%% Findings: A Turing machine can simulate a Turing machine with a different
%% tape alphabet, a multitape Turing machine, a random access machine, and a
%% nondeterministic Turing machine.
%%
%% Conclusion: The Turing machine is at least as powerful as any model of
%% computation (meeting the four requirements of input, output, memory, and
%% state control) that you can conceive.
%%
%% Perspectives: The power of the Turing machine in simulating other models
%% does not preclude the possibility of other models simulating Turing
%% machines.
\lipsum
