%% Main message: The Turing machine is a computational model which can
%% implement any ``algorithm''.
%%
Our intuitive notion of ``algorithm'' and ``computer'' is insufficient for proving mathematical statements about correctness or efficiency of computer programs.
A theoretical computer scientist needs a formal definition of a computer that can implement precisely defined algorithms, so that our claims can be backed up by mathematical proofs.
This part of the book defines an algorithm, proposes several models of computing devices, and presents the thesis that one of these models of computation is sufficient for implementing any algorithm.

We define an algorithm as a finite sequence of instructions, suggest a Turing machine as the simplest model of computation for executing any algorithm, and offer evidence that, although simple, the Turing machine is powerful enough to simulate any other reasonable model of computation.
Proving which problems are solvable (\autoref{prt:computability}) and which are solvable efficiently (\autoref{prt:complexity}) is only possible with the Turing machine clearly defined.
Although the Turing machine is sufficient to describe any computation, there are many, many other models of computation.
%% For more information, see...
